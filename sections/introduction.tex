\section{Introduction}
Automatic kinship recognition has numerous uses. For instance - as an aid in forensic investigations, automated photo library management, historical lineage and genealogical studies, social-media-based analysis, tragedies of missing children and human trafficking, and concerns about immigration and border patrol. Nonetheless, the challenges in such face-based tasks (\ie fine-grained classification in unconstrained settings), are only amplified in the kin-based problem sets, as the data exhibits a high degree of variability in pose, illumination, background, and clarity,  along with soft bio-metric target labels (\ie kinship), which only further exacerbates the challenges with consideration for the directional relationships. Hence, the usefulness brought by the  practical benefits of enhancing kinship-based technology is matched by the challenges posed by the problem of automatic kinship understanding. This motivated the launching of the \ac{rfiw} challenge series: a large-scale data challenge in support of multiple tasks with the aim to advance kinship recognition technologies. We intend for \ac{rfiw} to serve as a platform for expert and junior researchers to present and share thoughts in an open forum.

The \ac{fiw} dataset~\cite{robinson2016families, robinson2018visual, wang2017kinship}-- a large-scale, multi-task image set for kinship recognition-- supports the annual \ac{rfiw}.\footnote{\ac{fiw} project page, \href{https://web.northeastern.edu/smilelab/fiw/}{https://web.northeastern.edu/smilelab/fiw/}.} The aim of the \ac{rfiw} challenge is to bridge the gap between research-and-reality using its large scale, variation, and rich label information. This makes modern-day data-driven approaches possible, as has been seen since its release in 2016~\cite{AdvNet, ertugrul2017will, gao2019will, li2017kinnet, wu2018kinship}.

We summarize the evaluation protocols-- practical motivation, technical background, data splits, metrics, and benchmarks-- of the 2020 \ac{rfiw} challenge. Specifically, this manuscript serves as a white-paper of the \ac{rfiw} held in conjunction with the \ac{fg}. Additional and information supplemental on the challenge website.\footnote{\ac{rfiw}2020 webpage, \href{https://web.northeastern.edu/smilelab/rfiw2020/}{https://web.northeastern.edu/smilelab/rfiw2020/}.} 

The remainder of the paper is organized as follows. The three tasks that make-up \ac{rfiw}2020 are introduced separately (Section \ref{sec:kinver}, \ref{sec:trisubject}, and \ref{sec:search}). For each task, a clear problem statement, the intended use, data splits, task protocols (\ie evaluation settings and metrics), and benchmark results are provided. From there, we bring up the discussion (Section~\ref{sec:discussion}) on broader impacts and potential next steps. Then, we conclude (Section~\ref{sec:conclusion}).